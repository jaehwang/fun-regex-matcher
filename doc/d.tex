\section{Design}

\subsection{Matcher with Backtracking}

\begin{frame}
    \frametitle{설계 전략}

    \begin{itemize}
    \item State machine보다는 regular expression, regular language의 의미를
        드러내는 데 적합하게 설계한다.
    \item 설계, 구현, correctness 증명이 용이한 수학적 함수 형태로 설계 한다.
    \item Backtracking이 있는 matcher를 먼저 정의 한 후, 
    동적으로 DFA를 만드는 효과를 가지도록 수정한다.
    \end{itemize}
\end{frame}


\begin{frame}
    \frametitle{Definition of Matcher \M{} with Backtracking}

먼저 backtracking이 있는 함수 \M을 정의하자.

\[
\begin{array}{rcl}
  \rmatch{\e}{\e} & = & \T \nonumber \\
  \rmatch{\e}{c} & = & \F \nonumber \\
  \rmatch{\e}{\replus} & = & \rmatch{\e}{\rone}\vee\rmatch{\e}{\rtwo}
  \nonumber\\
  \rmatch{\e}{\retimes} & = & \rmatch{\e}{\rone}\wedge\rmatch{\e}{\rtwo} \nonumber \\
  \rmatch{\e}{\restar} & = & \T \nonumber \\[2ex]
  \rmatch{\cs}{re} & = & \mrm{if } \firstmatch(c,re)  \\
                   &   & \mrm{then } \bigvee_{re'\in\suffix(c,re)}\rmatch{s}{re'} \\
                   &   & \mrm{else } \F
\end{array}
\]

%where the first character match operator $\da$ is 
%  \[
%  c\da re  = \ifte{c\in\first(re)}{\T}{\F}
%  \]
\M의 정의에 사용한 \suffix, \firstmatch 등은 ``Auxiliary
Function''에서 정의한다.
\end{frame}

\subsection{Auxiliary Functions}\label{sec:aux}

\begin{frame}[shrink]
\frametitle{Auxiliary Function: \nullable}

정규 표현식이 $\e$를 가질 수 있는 지 판정하는 함수 \nullable.

\[
\begin{array}{rcl}
  \nullable(\e) & = & \mtt{true} \\
  \nullable(c) & = & \mtt{false} \\
  \nullable(\rone+\rtwo) & = & \nullable(\rone) \vee \nullable(\rtwo) \\
  \nullable(\rone\cdot\rtwo) & = & \nullable(\rone)\wedge\nullable(\rtwo) \\
  \nullable(re^{*}) & = & \mtt{true}
\end{array}
\]  

\begin{block}{Correctness of \nullable}
 \[
     \nullable(re) = \mtt{true} \Lra \e \in L(re)
 \]
\end{block}

\end{frame}

% \begin{frame}[shrink]
% \frametitle{Auxiliary Function: \first}
% 
% 정규 표현식으로 생성되는 스트링의 첫 글자 집합을 만드는 함수 \first.
% 
% \[
% \begin{array}{rcl}
%   \first(\e) & = & \varnothing \\
%   \first(c) & = & \{c\} \\
%   \first(\rone+\rtwo) & = & \first(\rone) \cup \first(\rtwo) \\
%   \first(\rone\cdot\rtwo) & = & \mrm{if } \nullable(\rone) \\
%                           &   & \mrm{then }\first(\rone)\cup\first(\rtwo) \\
%                           &   & \mrm{else }\first(\rone) \\
%   \first(re^{*}) & = & \first(re)
% \end{array}
% \]
% 
% \begin{block}{Correctness of \first}
%  \[
%   c\in\first(re)\Lra\exists s\mrm{ such that }\cs\in L(re).%\mrm{ for some string }s
%  \]
% \end{block}
% 
% \end{frame}
% 
\begin{frame}[shrink]
\frametitle{Auxiliary Function: \firstmatch}

글자 $c$로 시작하는 스트링이 정규 표현식 $re$로 생성되는 $L(re)$에 포함되는 지 알려주는 함수 \firstmatch.

\[
\begin{array}{rcl}
  \firstmatch(c,\e) & = & \F \\
  \firstmatch(c,c') & = & \ifte{c=c'}{\T}{\F} \\
  \firstmatch(c,\rone+\rtwo) & = & \firstmatch(c,\rone) \vee \firstmatch(c,\rtwo) \\
  \firstmatch(c,\rone\cdot\rtwo) & = & \mrm{if } \nullable(\rone) \\
                          &   & \mrm{then }\firstmatch(c,\rone)\vee\firstmatch(c,\rtwo) \\
                          &   & \mrm{else }\firstmatch(c,\rone) \\
  \firstmatch(c,re^{*}) & = & \firstmatch(c,re)
\end{array}
\]

\begin{block}{Correctness of \firstmatch}
 \[
  \firstmatch(c,re)=\T\Lra\exists s\mrm{ such that }\cs\in L(re).%\mrm{ for some string }s
 \]
\end{block}

\end{frame}

\begin{frame}[shrink]
\frametitle{Auxiliary Function: \suffix}

하나의 글자와 match되는 prefix를 뺀 나머지 정규 표현식을 계산하는 함수 \suffix.

    \[
    \begin{array}{rcl}
      \suffix(c,\e) & = & \emptyset \\
      \suffix(c,c)&=&\{\e\} \\
      \suffix(c,c')&=&\emptyset\mrm{ where }c\neq c' \\
      \suffix(c,\replus) &=& \suffix(c,\rone) \cup \suffix(c,\rtwo) \\

      \suffix(c,\rone\cdot \rtwo)& = & \{ r\cdot \rtwo \mid r \in
      \suffix(c,\rone) \} \cup \\
      &   & (\ifte{\nullable(\rone)}{\suffix(c,\rtwo)}{\emptyset}) \\

%      \suffix(c,\e\cdot re)& = &\suffix(c,re) \\
%      \suffix(c,c'\cdot re) &=& \ifte{c=c'}{\{re\}}{\emptyset} \\
%      \suffix(c,(\retimes)\cdot re)& =& \suffix(c,\rone\cdot(\rtwo\cdot re)) \\
%      \suffix(c,(\replus)\cdot re) &=& \suffix(c,\rone\cdot re)\cup\suffix(c,\rtwo\cdot re) \\
%      \suffix(c,\rone^{*}\cdot re)& =& \suffix(c,re)\cup \\
%      &  & (\ifte{c\da\rone^{*}}{\{re'\cdot re|re'\in \suffix(c,\rone^{*})\}}{\emptyset}) \\

      \suffix(c,re^{*}) &=& \mrm{if }\firstmatch(c,re) \\
                        & & \mrm{then }\{r\cdot re^{*}|r\in\suffix(c,re)\} \\
                        & & \mrm{else }\emptyset
    \end{array}
    \]

\begin{block}{Correctness of \suffix}
\[
    \bigcup_{r'\in \suffix(c,r)}L(r') = \{ s | c\cdot s \in L(r) \}
%    \begin{array}{rcl}
%    \suffix(c,re)=R\neq\emptyset & \Ra & L(c)(\bigcup_{re'\in R}L(re'))=L(re) \\
%     \suffix(c,re)=\emptyset & \Lra & c\notin\first(re)
%  \end{array}
  \]
\end{block}

\end{frame}

\subsection{Optimization - Simulating DFA}

\begin{frame}[shrink]
\frametitle{Thompson's NFA Simulation}

NFA의 한 상태에서 갈 수 있는 다음 상태들의 집합을 하나의 상태로 보는 것이
NFA를 DFA로 변환하는 기본 아이디어이다.

이 변환을 미리하지 않고 실행 중에 함으로써 backtracking이 일어나지 않는
regualar expression matcher \MM을 다음과 같이 정의할 수 있다.

    \[
        \MM  :  2^{\mbit{Regex}} \times \mbit{String} \ra \mbit{Bool}
    \]
\[
  \begin{array}{rcl}
    \rsmatch{s}{\emptyset} & = & \mtt{false} \\
    \rsmatch{\e}{R} & = & \bigvee_{re\in R}\nullable(re) \\
    \rsmatch{\cs}{R} & = & \mrm{if }\Firstmatch(c,R) \\
                     &   & \mrm{then }\rsmatch{s}{\bigcup_{re\in R}\suffix(c,re)} \\
                     &   & \mrm{else }{\F} 
  \end{array}
\]  
where $\Firstmatch$ is
  \begin{eqnarray}
%    c\dda\emptyset & = & \F \nonumber \\
%    c\dda R & = & \ifte{c\in\bigcup_{re\in R}\first(re)}{\T}{\F} \nonumber
    \Firstmatch(c,\emptyset) & = & \F \nonumber \\
    \Firstmatch(c, R) & = & \bigvee_{re\in R}\firstmatch(c,re) \nonumber
  \end{eqnarray}

\end{frame}
